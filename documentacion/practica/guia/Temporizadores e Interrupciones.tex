\documentclass[10pt,a4paper]{article}
\usepackage[spanish]{babel}
\usepackage[ansinew]{inputenc}
%\usepackage[hidelinks]{hyperref}
\usepackage[right=1.5cm,left=2.5cm,top=2cm,bottom=2cm,headsep=0cm,footskip=0.5cm]{geometry}
\usepackage{amsmath}
\usepackage{graphicx}
\usepackage{multicol}
\usepackage{enumerate}
\usepackage{enumerate}
\usepackage{caption}   	%para colocar los caption en las figuras dentro el ambiente minipage.
\usepackage{courier}
\usepackage{xcolor}
\usepackage{hyperref}

%\usepackage[hidelinks]{hyperref}
\title{Programaci�n de la EDU-CIAA en lenguaje C}
\author{}

\bibliographystyle{plain}
\hypersetup{ %
colorlinks=flase,linkbordercolor=black,linkcolor=blue,pdfborderstyle={/S/U/W 1}
}
%pdfborder = {0 0 0}
\begin{document}

%\begin{titlepage}

\setlength{\unitlength}{0.2cm}\hypersetup{colorlinks=false,linkbordercolor=red,linkcolor=green,pdfborderstyle={/S/U/W 1}}
 %Especificar unidad de trabajo

\begin{multicols}{2}

\begin{flushleft}
    \footnotesize{ {\textbf{Curso Introductorio} }\\5ta Escuela de Sistemas Embebidos\\Tucum�n - Horco Molle 2015\\{RUSE - ACSE}}
\end{flushleft}
\columnbreak
\bigskip
\begin{flushright}
    \includegraphics[width=1.5in]{Imagenes/logo_ruse} \\ \includegraphics[width=1.5in]{Imagenes/acse}\\
\end{flushright}
\end{multicols}
\vspace{0.5cm}

\begin{center}
    {
    \textbf{\Large Manejo de Timer e Interrupciones para el microcontrolador LPC43XX con la biblioteca LPCOpen}\\[1.5cm]
	}
\end{center}

%\newpage
\vspace{1cm}

En este trabajo pr�ctico se puede utilizar el Timer de Interrupciones Repetitivas (pag. 1073 del User Manual) cuyas funciones est�n implementadas en el archivo \textit{ritimer\_18xx\_43xx.c}, de la biblioteca LPCOpen.\\

Tambi�n disponemos de una funci�n de inicializaci�n del temporizador:\\

\href{http://docs.lpcware.com/lpcopen/v1.03/group___r_i_t__18_x_x__43_x_x.html#ga9a8dc91573ba6d4556d787a62a8a3f7e}{\texttt{Chip\_RIT\_Init(LPC\_RITIMER\_T* pRITimer);}} \\

A la misma hay que pasarle como par�metro la direcci�n base del perif�rico RIT definida ya en \texttt{chip\_lpc 43xx.h} como LPC\_RITIMER.\\

Se debe configurar el intervalo (en ms.) de interrupci�n empleando la funci�n:\\

\href{http://docs.lpcware.com/lpcopen/v1.03/group___r_i_t__18_x_x__43_x_x.html#ga9473901c9e5ba4867c14597473cacf2b}{\texttt{Chip\_RIT\_SetTimer(LPC\_RITIMER\_T* pRITimer, uint32\_T intervalo);}}\\

Para escribir el c�digo de servicio de la interrupci�n, se debe generar una funci�n:\\

\texttt{void NombreDeLaIRutinaDeServicio(void);} \\

 y enlazar ese nombre en el vector de interrupciones definido en \texttt{vector.c}.\\
 
Para habilitar la interrupci�n se utiliza la funci�n:\\

\texttt{NVIC\_EnableIRQ(IRQn\_Type IRQn);}  definida en \texttt{core\_cm4.h}.\\

Para borrar el flag de interrupci�n pendiente, existe en \texttt{ritimer\_18xx\_43xx.c} una funci�n para esto:\\

\href{http://docs.lpcware.com/lpcopen/v1.03/group___r_i_t__18_x_x__43_x_x.html#ga16ba8104cee04723ddd751b99655f01d}{\texttt{Chip\_RIT\_ClearInt();}}

\end{document}

