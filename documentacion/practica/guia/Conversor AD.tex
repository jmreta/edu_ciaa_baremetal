\documentclass[10pt,a4paper]{article}
\usepackage[spanish]{babel}
\usepackage[ansinew]{inputenc}
%\usepackage[hidelinks]{hyperref}
\usepackage[right=1.5cm,left=2.5cm,top=2cm,bottom=2cm,headsep=0cm,footskip=0.5cm]{geometry}
\usepackage{amsmath}
\usepackage{graphicx}
\usepackage{multicol}
\usepackage{enumerate}
\usepackage{enumerate}
\usepackage{caption}   	%para colocar los caption en las figuras dentro el ambiente minipage.
\usepackage{courier}
\usepackage{xcolor}
\usepackage{hyperref}

%\usepackage[hidelinks]{hyperref}
\title{Programaci�n de la EDU-CIAA en lenguaje C}
\author{}

\bibliographystyle{plain}
\hypersetup{ %
colorlinks=flase,linkbordercolor=black,linkcolor=blue,pdfborderstyle={/S/U/W 1}
}
%pdfborder = {0 0 0}
\begin{document}

%\begin{titlepage}

\setlength{\unitlength}{0.2cm}\hypersetup{colorlinks=false,linkbordercolor=red,linkcolor=green,pdfborderstyle={/S/U/W 1}}
 %Especificar unidad de trabajo

\begin{multicols}{2}

\begin{flushleft}
    \footnotesize{ {\textbf{Curso Introductorio} }\\5ta Escuela de Sistemas Embebidos\\Tucum�n - Horco Molle 2015\\{RUSE - ACSE}}
\end{flushleft}
\columnbreak
\bigskip
\begin{flushright}
    \includegraphics[width=1.5in]{Imagenes/logo_ruse} \\ \includegraphics[width=1.5in]{Imagenes/acse}\\
\end{flushright}
\end{multicols}
\vspace{0.5cm}

\begin{center}
    {
    \textbf{\Large Manejo del Conversor Anal�gico-Digital para el
        microcontrolador LPC43XX con la biblioteca LPCOpen}\\[1.5cm]
	}
\end{center}

%\newpage
\vspace{1cm}
En este trabajo, adem�s de todas las funciones empleadas hasta este momento de GPIO, RITimer e IRQ, vamos a incorporar las funciones de uso del conversor Anal�gico a Digital (pag 1327 del User Manual ).\\

En primer lugar se debe configurar la System Control Unit (SCU): Algunos pines soportan el multiplexado de funciones digitales y anal�gicas, sin embargo, todas las entradas y salidas anal�gicas del ADC y DAC est�n adem�s ruteadas a pines de funci�n anal�gica sin necesidad de multiplexado.\\

Lo que si es necesario, es indicarle que vamos a utilizar el conversor AD mediante la funci�n:\\

\href{http://docs.lpcware.com/lpcopen/v1.03/group___s_c_u__18_x_x__43_x_x.html#ga198684376606623332684569065a6d27}{\texttt{Chip\_SCU\_ADC\_Channel\_Config(uint32\_t ADC\_ID, uint8\_t channel);}}\\

Para despu�s de eso, utilizar las funciones de LPOpen para el manejo del
conversor \href{http://docs.lpcware.com/lpcopen/v1.03/group___a_d_c__18_x_x__43_x_x.html}{AD}, incorporadas en  \texttt{\textit{adc\_18xx\_43xx.h}} :\\

\href{http://docs.lpcware.com/lpcopen/v1.03/group___a_d_c__18_x_x__43_x_x.html#ga9c3e36e83d27f02ccabcba7e5a9befd0}{\texttt{Chip\_ADC\_Init(LPC\_ADC\_T *pADC, ADC\_Clock\_Setup\_T* ADCSetup );}}\\

\href{http://docs.lpcware.com/lpcopen/v1.03/group___a_d_c__18_x_x__43_x_x.html#gac16bb5474c843f941966e829d59d854e}{\texttt{Chip\_ADC\_EnableChannel(LPC\_ADC\_T *pADC, ADC\_CHANNEL\_T channel,FunctionalState NewState)}}\\

\href{http://docs.lpcware.com/lpcopen/v1.03/group___a_d_c__18_x_x__43_x_x.html#ga7a349732e44df642b1e86c63f5289bc3}{\texttt{Chip\_ADC\_SetStartMode(LPC\_ADC\_T *pADC, ADC\_START\_MODE\_T mode,ADC\_EDGE\_CFG\_T EdgeOption)}}\\

\href{http://docs.lpcware.com/lpcopen/v1.03/group___a_d_c__18_x_x__43_x_x.html#ga14831abe4fd86e0617cedccca77b19cf}{\texttt{Chip\_ADC\_ReadStatus(LPC\_ADC\_T *pADC, uint8\_t channel, uint32\_t StatusType)}}\\

\href{http://docs.lpcware.com/lpcopen/v1.03/group___a_d_c__18_x_x__43_x_x.html#gad413251b83a9c9940569ac3db31d3dfa}{\texttt{Chip\_ADC\_ReadValue(LPC\_ADC\_T *pADC, uint8\_t channel, uint16\_t *data)}}\\



\end{document}

