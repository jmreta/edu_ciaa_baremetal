\documentclass[10pt,a4paper]{article}
\usepackage[spanish]{babel}
\usepackage[ansinew]{inputenc}
%\usepackage[hidelinks]{hyperref}
\usepackage[right=1.5cm,left=2.5cm,top=2cm,bottom=2cm,headsep=0cm,footskip=0.5cm]{geometry}
\usepackage{amsmath}
\usepackage{graphicx}
\usepackage{multicol}
\usepackage{enumerate}
\usepackage{enumerate}
\usepackage{caption}   	%para colocar los caption en las figuras dentro el ambiente minipage.
\usepackage{courier}
\usepackage{xcolor}
\usepackage{hyperref}

%\usepackage[hidelinks]{hyperref}
\title{Programaci�n de la EDU-CIAA en lenguaje C}
\author{}

\bibliographystyle{plain}
\hypersetup{ %
colorlinks=flase,linkbordercolor=black,linkcolor=blue,pdfborderstyle={/S/U/W 1}
}
%pdfborder = {0 0 0}
\begin{document}

%\begin{titlepage}

\setlength{\unitlength}{0.2cm}\hypersetup{colorlinks=false,linkbordercolor=red,linkcolor=green,pdfborderstyle={/S/U/W 1}}
 %Especificar unidad de trabajo

\begin{multicols}{2}

\begin{flushleft}
    \footnotesize{ {\textbf{Curso Introductorio} }\\5ta Escuela de Sistemas Embebidos\\Tucum�n - Horco Molle 2015\\{RUSE - ACSE}}
\end{flushleft}
\columnbreak
\bigskip
\begin{flushright}
    \includegraphics[width=1.5in]{Imagenes/logo_ruse} \\ \includegraphics[width=1.5in]{Imagenes/acse}\\
\end{flushright}
\end{multicols}
\vspace{0.5cm}

\begin{center}
    {
    \textbf{\Large Manejo de Timer e Interrupciones para el microcontrolador LPC 43XX con la biblioteca LPCOpen}\\[1.5cm]
	}
\end{center}

%\newpage
\vspace{1cm}


\section{Temporizador e Interrupciones}

\subsection{Consignas}
\begin{enumerate}[ a) ]

\item Dise�e e implemente un firmware sobre la EDU-CIAA que encienda de a un led por vez y de manera secuencial. El tiempo de encendido de cada led ser 250ms. Se deber� temporizar mediante interrupciones sin usar funciones de retardo por software.


\item Dise�e e implemente un firmware sobre la EDU-CIAA que haga parpadear un led con un periodo de 250 ms. El sistema debe permitir seleccionar uno de entre 4 de los leds disponibles empleando una tecla para cada led.\\
\begin{itemize}
\item Tec 1: Selecciona LED RGB (uno de los tres colores)
\item Tec 2: Selecciona LED 1.
\item Tec 3: Selecciona LED 2.
\item Tec 4: Selecciona LED 3.
\end{itemize}


\item \textbf{Consigna 2.3}: 

Incorpore al ejercicio anterior la funcionalidad de variar el periodo de parpadeo del led activo.\\

\begin{itemize}
\item Tec 1: Selecciona el LED a la izquierda del actual.
\item Tec 2: Selecciona LED a la derecha del actual.
\item Tec 3: Disminuye el periodo de parpadeo.
\item Tec 4: Aumenta el periodo de parpadeo.
\end{itemize}

\end{enumerate}


%\begin{center}
\section{\large Generaci�n de Se�ales Anal�gicas (D/A)}
%\end{center}

\begin{itemize}

\item \textbf{Consigna 3.1}: Dise�e e implemente un firmware sobre la EDU-CIAA que genera una se�al tipo diente de sierra de periodo 100 ms y excursi�n de 0 a 3V.

\item \textbf{Cosigna 3.2}: Incorpore al ejercicio anterior la funcoinalidad de variariar el periodo y la amplitud de la se�al. \\

\begin{itemize}
\item Tec 1: Aumenta la amplitud de la se�al. 
\item Tec 2: Disminuye la amplitu de la se�al.
\item Tec 3: Aumenta el periodo de la se�al.
\item Tec 4: disminuye el periodo de la se�al. 
\end{itemize}

\end{itemize}


%\begin{center}
\section{\large Adquisici�n de datos}

\subsection{Consignas}
\begin{enumerate}[ a) ]

\item Dise�e e implemente un Firmware sobre la EDU-CIAA que permita adquirir una se�al anal �gica de excursi�n entre 0 y 3.3V, presente en el pin CH1 del conector P1. El sistema debe encender el led rojo si la se�al toma su valor m�ximo (>1020 d) y led verde si la se�al toma su valor m�nimo (< 5 d). Puede resolverlo por pooling.\\



\item  Incorpore al ejercicio anterior la funcionalidad de variar los umbrales m�ximo y m�nimo, y ajuste la frecuencia de conversi�n en 100Hz empleando timer e interrupciones. La lectura del conversor, tambi�n debe hacerse por interrupciones.\\

\begin{itemize}
\item Tec 1: Aumenta el valor del umbral.
\item Tec 2: Disminuye el valor del umbral.
\end{itemize}

\end{enumerate}

%\begin{center}
\section{\large Transmisi�n de datos adquiridos a trav�s del Puerto Serie}
%\end{center}

\subsection{Consignas}
\begin{enumerate}[ a) ]

\item Dise�e e implemente un firmware sobre la EDU-CIAA que env�e por el puerto serie la cadena \textit{Hola Mundo} a tasa de transferencia de 9600 baudios. La cadena debe enviarse cada vez que el usuario presiona la tecla 1. 

\item Dise�e e implemente un firmware sobre la EDU-CIAA que lea el puerto serie a la espera del caracter ascii 'a'.
En respuesta el sistema debe enviar por el mismo puerto la cadena "Hola Mundo" y cambiar el estado del LED 2 de la placa.

\item Dise�e e implemente un firmware sobre la EDU-CIAA que lea el puerto serie a la espera del caracteres ascii.
El sistema solo responder� a los car�cteres 'a', 'r' y 'v' seg�n el siguiente detalle:

\begin{itemize}
\item Caracter 'a': Se debe cambiar el estado del LED 1 y enviar por el puerto serie la cadena: \textit{Hola Mundo}
\item Caracter 'r': Se debe cambiar el estado del LED 2 y enviar por el puerto serie la cadena: \textit{Hola Mundo}
\item Caracter 'v': Se debe cambiar el estado del LED 3 y enviar por el puerto serie la cadena: \textit{Hola Mundo}
\end{itemize}


\item Dise�e e implemente un firmware sobre la EDU-CIAA que env�e por el puerto serie, cada vez qu se presiona la tecla 1, el valor de un contador de 256 cuentas. El valor del contador debe modificarse mediante las teclas 4 y 5.

\item Dise�e e implemente un firmware sobre la EDU-CIAA que permita adquirir una se�al anal�gica de excursi�n entre 0 y 3.3V,  presente en el CH1. El sistema debe enviar por el puerto serie una cadena de caracteres con el valor en decimal del dato convertido.\\

\end{enumerate}

%\begin{center}
\section{\large Manejo de perif�ricos con POSIX}
%\end{center}

Repita las consignas 1.1, 1.2 y 2.1 usando las funciones POSIX definidas en el Firmware de la CIAA.

\section{\large Ejercicios Integradores}

Realice un firmware sobre la EDU-CIAA, que permita adquirir una se�al anal�gica acondicionada entre 0 y 3,3 V disponible en el pin CH1 de P1 a una frecuencia de 100 hz.\\

La se�al se debe reproducir por el DA, luego de alg�n procesamiento, por ejemplo ganancia, filtrado, recorte, etc. El procesamiento debe controlarse mediante comandos provenientes del puerto serie que permiten encender, apagar y modificar cada una de las rutinas de procesamiento disponibles.\\
\underline{Nota:} \href{http://www-users.cs.york.ac.uk/~fisher/mkfilter/}{Link Calculo de Filtros.}

%\bibliography{SAPS_bibliografia}
\end{document}
