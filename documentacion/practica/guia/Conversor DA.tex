\documentclass[10pt,a4paper]{article}
\usepackage[spanish]{babel}
\usepackage[ansinew]{inputenc}
%\usepackage[hidelinks]{hyperref}
\usepackage[right=1.5cm,left=2.5cm,top=2cm,bottom=2cm,headsep=0cm,footskip=0.5cm]{geometry}
\usepackage{amsmath}
\usepackage{graphicx}
\usepackage{multicol}
\usepackage{enumerate}
\usepackage{enumerate}
\usepackage{caption}   	%para colocar los caption en las figuras dentro el ambiente minipage.
\usepackage{courier}
\usepackage{xcolor}
\usepackage{hyperref}

%\usepackage[hidelinks]{hyperref}
\title{Programaci�n de la EDU-CIAA en lenguaje C}
\author{}

\bibliographystyle{plain}
\hypersetup{ %
colorlinks=flase,linkbordercolor=black,linkcolor=blue,pdfborderstyle={/S/U/W 1}
}
%pdfborder = {0 0 0}
\begin{document}

%\begin{titlepage}

\setlength{\unitlength}{0.2cm}\hypersetup{colorlinks=false,linkbordercolor=red,linkcolor=green,pdfborderstyle={/S/U/W 1}}
 %Especificar unidad de trabajo

\begin{multicols}{2}

\begin{flushleft}
    \footnotesize{ {\textbf{Curso Introductorio} }\\5ta Escuela de Sistemas Embebidos\\Tucum�n - Horco Molle 2015\\{RUSE - ACSE}}
\end{flushleft}
\columnbreak
\bigskip
\begin{flushright}
    \includegraphics[width=1.5in]{Imagenes/logo_ruse} \\ \includegraphics[width=1.5in]{Imagenes/acse}\\
\end{flushright}
\end{multicols}
\vspace{0.5cm}

\begin{center}
    {
    \textbf{\Large Manejo del Conversor Digital-Anal�gico para el
        microcontrolador LPC43XX con la biblioteca LPCOpen}\\[1.5cm]
	}
\end{center}

%\newpage
\vspace{1cm}

En este trabajo, adem�s de todas las funciones empleadas hasta este momento de GPIO, RITimer e IRQ, vamos a incorporar las funciones de uso del conversor Digital a Anal�gico (pag 1350 del User Manual).\\

Primero hay que configurar la System Control Unit (SCU): Algunos pines soportan el multiplexado de funciones digitales y anal�gicas, sin embargo, todas las entradas y salidas anal�gicas del ADC y DAC est�n adem�s ruteadas a pines de funci�n anal�gica sin necesidad de multiplexado.\\

Lo que si es necesario, es indicarle que vamos a utilizar el conversor DA mediante la funci�n: \\

\href{http://docs.lpcware.com/lpcopen/v1.03/group___s_c_u__18_x_x__43_x_x.html#ga5dd513d87e5d14f80544f59692536a92}{\texttt{Chip\_SCU\_DAC\_Analog\_Config();}}\\

Para despu�s de eso, utilizar las funciones de LPOpen para el manejo del conversor \href{http://docs.lpcware.com/lpcopen/v1.03/group___d_a_c__18_x_x__43_x_x.html}{DA}, incorporadas en \texttt{\textit{dac\_18xx\_43xx.h}}:\\

\href{http://docs.lpcware.com/lpcopen/v1.03/group___d_a_c__18_x_x__43_x_x.html#ga677c6f03e4ea92656c4cb3497fbb4a1b}{\texttt{Chip\_DAC\_Init(LPC\_DAC\_T *pDAC);}}\\

\href{http://docs.lpcware.com/lpcopen/v1.03/group___d_a_c__18_x_x__43_x_x.html#ga4cfcf4f22b8d719d9cb9947a4e223c32}{\texttt{Chip\_DAC\_UpdateValue((LPC\_DAC\_T *pDAC, uint32\_t dac\_value);}}\\



\end{document}

