\documentclass[10pt,a4paper]{article}
\usepackage[spanish]{babel}
\usepackage[ansinew]{inputenc}
%\usepackage[hidelinks]{hyperref}
\usepackage[right=1.5cm,left=2.5cm,top=2cm,bottom=2cm,headsep=0cm,footskip=0.5cm]{geometry}
\usepackage{amsmath}
\usepackage{graphicx}
\usepackage{multicol}
\usepackage{enumerate}
\usepackage{enumerate}
\usepackage{caption}   	%para colocar los caption en las figuras dentro el ambiente minipage.
\usepackage{courier}
\usepackage{xcolor}
\usepackage{hyperref}

%\usepackage[hidelinks]{hyperref}
\title{Programaci�n de la EDU-CIAA en lenguaje C}
\author{}

\bibliographystyle{plain}
\hypersetup{ %
colorlinks=flase,linkbordercolor=black,linkcolor=blue,pdfborderstyle={/S/U/W 1}
}
%pdfborder = {0 0 0}
\begin{document}

%\begin{titlepage}

\setlength{\unitlength}{0.2cm}\hypersetup{colorlinks=false,linkbordercolor=red,linkcolor=green,pdfborderstyle={/S/U/W 1}}
 %Especificar unidad de trabajo

\begin{multicols}{2}

\begin{flushleft}
    \footnotesize{ {\textbf{Curso Introductorio} }\\5ta Escuela de Sistemas Embebidos\\Tucum�n - Horco Molle 2015\\{RUSE - ACSE}}
\end{flushleft}
\columnbreak
\bigskip
\begin{flushright}
    \includegraphics[width=1.5in]{Imagenes/logo_ruse} \\ \includegraphics[width=1.5in]{Imagenes/acse}\\
\end{flushright}
\end{multicols}
\vspace{0.5cm}

\begin{center}
    {
    \textbf{\Large Manejo de Puertos en el microcontrolador LPC43XX con la biblioteca LPCOpen}\\[1.5cm]
}
\end{center}





Para la realizaci�n de todos los trabajos pr�cticos de este curso, se utilizar� la librer�a LPCOpen para el acceso a los perif�ricos espec�ficos del LPC4337.\\

Esta biblioteca, se incluye en el Firmware de la CIAA, en:\\
 $...CIAA\backslash Firmware\backslash externals\backslash drivers\backslash cortexM4\backslash lpc43xx$\\

Es destacable, que para emplear un procesador diferente, se debe cambiar esta bibioteca, por  ejemplo, para trabajar con la CIAA\_FSL\ (CIAA Freescale), es necesario cargar los drivers que se encuentran en:\\
 $...CIAA\backslash Firmware\backslash externals\backslash drivers\backslash cortexM4\backslash k60\_120 $.\\

\section{Puertos de Salida: LEDS}

Para el manejo de puertos de entrada salida de uso general (GPIO), el archivo gpio\_18xx\_43xx.c, contiene todos los drivers para manejo del mismo. Ver documentaci�n de \href{http://docs.lpcware.com/lpcopen/v1.03/group___g_p_i_o__18_x_x__43_x_x.html}{LPCOpen}.\\

Para configurar un GPIO, lo primero es llamar a la funci�n 
\href{http://docs.lpcware.com/lpcopen/v1.03/group___g_p_i_o__18_x_x__43_x_x.html#gaeaca39372c8ff9f288243a20dd2259ce}{\texttt{Chip\_GPIO\_Init(LPC\_GPIO\_T * pGPIO)}}, a la hay que pasarle como par�metro la direcci�n base del perif�rico GPIO definida ya en chip\_lpc 43xx.h como LPC\_GPIO\_PORT.\\

Luego hay que configurar la System Control Unit (SCU), para indicarle las caracter�sticas el�ctricas de cada pin empleado  y remapearlos como puertos GPIO. Hay que recordar que en este procesador, se puede elegir entre varias funciones disponibles para cada pin (ver Tabla 189 en la p�gina 397 del User Manual):\\

\texttt{Chip\_SCU\_PinMux(2,0,MD\_PUP,FUNC4); /* mapea P2\_0  en GPIO5[0], LED0R y habilita el pull up*/}\\

\texttt{Chip\_SCU\_PinMux(2,1,MD\_PUP,FUNC4); /* mapea P2\_1  en GPIO5[1], LED0G y habilita el pull up */}

\texttt{Chip\_SCU\_PinMux(2,2,MD\_PUP,FUNC4); /* mapea P2\_2  en GPIO5[2], LED0B y habilita el pull up */}

\texttt{Chip\_SCU\_PinMux(2,10,MD\_PUP,FUNC0); /* remapea P2\_10 en GPIO0[14], LED1 y habilita el pull up */}

\texttt{Chip\_SCU\_PinMux(2,11,MD\_PUP,FUNC0); /* remapea P2\_11 en GPIO1[11], LED2 y habilita el pull up */}

\texttt{Chip\_SCU\_PinMux(2,12,MD\_PUP,FUNC0); /* remapea P2\_12 en GPIO1[12], LED3 y habilita el pull up */}\\

A continuaci�n, se debe seleccionar el modo (entrada o salida) de cada pin con la funci�n:\\

\href{http://docs.lpcware.com/lpcopen/v1.03/group___g_p_i_o__18_x_x__43_x_x.html#ga30d84fb97b47e0a8dc3a249cd77f34f7}{\texttt{Chip\_GPIO\_SetDir(LPC\_GPIO\_T * pGPIO, uint8\_t portNum,uitn32\_t portValue,uint8\_t out);}}\\

Para setear y resetear los pines, existen numerosas funciones, entre ellas:\\
\begin{flushleft}

\texttt{Chip\_GPIO\_ClearValue();}\\
\texttt{Chip\_GPIO\_SetValue();}\\
\texttt{Chip\_GPIO\_SetPinOutLow();}\\
\texttt{Chip\_GPIO\_SetPinOutHigh();}\\
\texttt{Chip\_GPIO\_SetPortOutHigh();}\\
\texttt{Chip\_GPIO\_SetPinToggle();}\\
\texttt{Chip\_GPIO\_SetPortToggle();}\\
\end{flushleft}

A las que siempre hay que pasarles como par�metro la direcci�n base del perif�rico GPIO (LPC\_GPIO\_PORT), el n�mero de puerto y el bit a modificar.
En caso de las funciones que hacen referencia a un solo GPIO (contienen la palabra �Pin�) se le indica el n�mero de bit del puerto que se desea modificar.\\

En caso de funciones que acceden a todo el puerto (Port), se le pasa una m�scara del tipo uint32\_T con los bits a modificar en 1.


\section{Puertos de Entrada: Pulsadores}

Para trabajar con los puertos de salida disponibles en la placa se deben configurar las entradas digitales correspondientes a las teclas.\\
\begin{flushleft}

\texttt{Chip\_SCU\_PinMux(1,0,MD\_PUP|MD\_EZI|MD\_ZI,FUNC0);/* mapea P1\_0  en GPIO 0[4], SW1 */}\\
\texttt{Chip\_SCU\_PinMux(1,0,MD\_PUP|MD\_EZI|MD\_ZI,FUNC0);/* mapea P1\_0  en GPIO 0[4], SW1 */}\\
\texttt{Chip\_SCU\_PinMux(1,1,MD\_PUP|MD\_EZI|MD\_ZI,FUNC0);/* mapea P1\_1  en GPIO 0[8], SW2 */}\\
\texttt{Chip\_SCU\_PinMux(1,2,MD\_PUP|MD\_EZI|MD\_ZI,FUNC0);/* mapea P1\_2  en GPIO 0[9], SW3 */}\\
\texttt{Chip\_SCU\_PinMux(1,6,MD\_PUP|MD\_EZI|MD\_ZI,FUNC0);/* mapea P1\_6  en GPIO 1[9], SW4 */}\\
\end{flushleft}

Adem�s, habilita para cada pin el buffer de entrada y deshabilita el filtro de glitch (ver figura 41 del User Manual).
Es interesante como re realiza esta acci�n en el driver de GPIO del Firmware de la CIAA: �ciaaDriverDio.c�, que se encuentra en %�CIAA\Firmware\modules\drivers\cortexM4\lpc43xx\lpc4337\src
\\
Luego setear estos pines como entrada (\href{http://docs.lpcware.com/lpcopen/v1.03/group___g_p_i_o__18_x_x__43_x_x.html#ga30d84fb97b47e0a8dc3a249cd77f34f7}{\texttt{Chip\_GPIO\_SetDir}}), y para leer, se pueden utilizar las funciones:\\
\begin{flushleft}
\texttt{Chip\_GPIO\_ReadValue()}\\
\texttt{Chip\_GPIO\_ReadPortBit()}
\end{flushleft}

\end{document}